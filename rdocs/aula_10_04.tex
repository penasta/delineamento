% Options for packages loaded elsewhere
\PassOptionsToPackage{unicode}{hyperref}
\PassOptionsToPackage{hyphens}{url}
%
\documentclass[
  a4paper]{article}
\usepackage{amsmath,amssymb}
\usepackage{lmodern}
\usepackage{iftex}
\ifPDFTeX
  \usepackage[T1]{fontenc}
  \usepackage[utf8]{inputenc}
  \usepackage{textcomp} % provide euro and other symbols
\else % if luatex or xetex
  \usepackage{unicode-math}
  \defaultfontfeatures{Scale=MatchLowercase}
  \defaultfontfeatures[\rmfamily]{Ligatures=TeX,Scale=1}
\fi
% Use upquote if available, for straight quotes in verbatim environments
\IfFileExists{upquote.sty}{\usepackage{upquote}}{}
\IfFileExists{microtype.sty}{% use microtype if available
  \usepackage[]{microtype}
  \UseMicrotypeSet[protrusion]{basicmath} % disable protrusion for tt fonts
}{}
\makeatletter
\@ifundefined{KOMAClassName}{% if non-KOMA class
  \IfFileExists{parskip.sty}{%
    \usepackage{parskip}
  }{% else
    \setlength{\parindent}{0pt}
    \setlength{\parskip}{6pt plus 2pt minus 1pt}}
}{% if KOMA class
  \KOMAoptions{parskip=half}}
\makeatother
\usepackage{xcolor}
\usepackage[margin=1in]{geometry}
\usepackage{color}
\usepackage{fancyvrb}
\newcommand{\VerbBar}{|}
\newcommand{\VERB}{\Verb[commandchars=\\\{\}]}
\DefineVerbatimEnvironment{Highlighting}{Verbatim}{commandchars=\\\{\}}
% Add ',fontsize=\small' for more characters per line
\usepackage{framed}
\definecolor{shadecolor}{RGB}{248,248,248}
\newenvironment{Shaded}{\begin{snugshade}}{\end{snugshade}}
\newcommand{\AlertTok}[1]{\textcolor[rgb]{0.94,0.16,0.16}{#1}}
\newcommand{\AnnotationTok}[1]{\textcolor[rgb]{0.56,0.35,0.01}{\textbf{\textit{#1}}}}
\newcommand{\AttributeTok}[1]{\textcolor[rgb]{0.77,0.63,0.00}{#1}}
\newcommand{\BaseNTok}[1]{\textcolor[rgb]{0.00,0.00,0.81}{#1}}
\newcommand{\BuiltInTok}[1]{#1}
\newcommand{\CharTok}[1]{\textcolor[rgb]{0.31,0.60,0.02}{#1}}
\newcommand{\CommentTok}[1]{\textcolor[rgb]{0.56,0.35,0.01}{\textit{#1}}}
\newcommand{\CommentVarTok}[1]{\textcolor[rgb]{0.56,0.35,0.01}{\textbf{\textit{#1}}}}
\newcommand{\ConstantTok}[1]{\textcolor[rgb]{0.00,0.00,0.00}{#1}}
\newcommand{\ControlFlowTok}[1]{\textcolor[rgb]{0.13,0.29,0.53}{\textbf{#1}}}
\newcommand{\DataTypeTok}[1]{\textcolor[rgb]{0.13,0.29,0.53}{#1}}
\newcommand{\DecValTok}[1]{\textcolor[rgb]{0.00,0.00,0.81}{#1}}
\newcommand{\DocumentationTok}[1]{\textcolor[rgb]{0.56,0.35,0.01}{\textbf{\textit{#1}}}}
\newcommand{\ErrorTok}[1]{\textcolor[rgb]{0.64,0.00,0.00}{\textbf{#1}}}
\newcommand{\ExtensionTok}[1]{#1}
\newcommand{\FloatTok}[1]{\textcolor[rgb]{0.00,0.00,0.81}{#1}}
\newcommand{\FunctionTok}[1]{\textcolor[rgb]{0.00,0.00,0.00}{#1}}
\newcommand{\ImportTok}[1]{#1}
\newcommand{\InformationTok}[1]{\textcolor[rgb]{0.56,0.35,0.01}{\textbf{\textit{#1}}}}
\newcommand{\KeywordTok}[1]{\textcolor[rgb]{0.13,0.29,0.53}{\textbf{#1}}}
\newcommand{\NormalTok}[1]{#1}
\newcommand{\OperatorTok}[1]{\textcolor[rgb]{0.81,0.36,0.00}{\textbf{#1}}}
\newcommand{\OtherTok}[1]{\textcolor[rgb]{0.56,0.35,0.01}{#1}}
\newcommand{\PreprocessorTok}[1]{\textcolor[rgb]{0.56,0.35,0.01}{\textit{#1}}}
\newcommand{\RegionMarkerTok}[1]{#1}
\newcommand{\SpecialCharTok}[1]{\textcolor[rgb]{0.00,0.00,0.00}{#1}}
\newcommand{\SpecialStringTok}[1]{\textcolor[rgb]{0.31,0.60,0.02}{#1}}
\newcommand{\StringTok}[1]{\textcolor[rgb]{0.31,0.60,0.02}{#1}}
\newcommand{\VariableTok}[1]{\textcolor[rgb]{0.00,0.00,0.00}{#1}}
\newcommand{\VerbatimStringTok}[1]{\textcolor[rgb]{0.31,0.60,0.02}{#1}}
\newcommand{\WarningTok}[1]{\textcolor[rgb]{0.56,0.35,0.01}{\textbf{\textit{#1}}}}
\usepackage{longtable,booktabs,array}
\usepackage{calc} % for calculating minipage widths
% Correct order of tables after \paragraph or \subparagraph
\usepackage{etoolbox}
\makeatletter
\patchcmd\longtable{\par}{\if@noskipsec\mbox{}\fi\par}{}{}
\makeatother
% Allow footnotes in longtable head/foot
\IfFileExists{footnotehyper.sty}{\usepackage{footnotehyper}}{\usepackage{footnote}}
\makesavenoteenv{longtable}
\usepackage{graphicx}
\makeatletter
\def\maxwidth{\ifdim\Gin@nat@width>\linewidth\linewidth\else\Gin@nat@width\fi}
\def\maxheight{\ifdim\Gin@nat@height>\textheight\textheight\else\Gin@nat@height\fi}
\makeatother
% Scale images if necessary, so that they will not overflow the page
% margins by default, and it is still possible to overwrite the defaults
% using explicit options in \includegraphics[width, height, ...]{}
\setkeys{Gin}{width=\maxwidth,height=\maxheight,keepaspectratio}
% Set default figure placement to htbp
\makeatletter
\def\fps@figure{htbp}
\makeatother
\setlength{\emergencystretch}{3em} % prevent overfull lines
\providecommand{\tightlist}{%
  \setlength{\itemsep}{0pt}\setlength{\parskip}{0pt}}
\setcounter{secnumdepth}{-\maxdimen} % remove section numbering
\ifLuaTeX
  \usepackage{selnolig}  % disable illegal ligatures
\fi
\IfFileExists{bookmark.sty}{\usepackage{bookmark}}{\usepackage{hyperref}}
\IfFileExists{xurl.sty}{\usepackage{xurl}}{} % add URL line breaks if available
\urlstyle{same} % disable monospaced font for URLs
\hypersetup{
  hidelinks,
  pdfcreator={LaTeX via pandoc}}

\author{}
\date{\vspace{-2.5em}}

\begin{document}

\begin{center}
{\Large
  DEPARTAMENTO DE ESTATÍSTICA} \\
\vspace{0.5cm}
\begin{figure}[!t]
\centering
\includegraphics[width=9cm, keepaspectratio]{logo-UnB.eps}
\end{figure}
\vskip 1em
{\large
  17 abril 2023}
\vskip 3em
{\LARGE
  \textbf{Lista 1}} \\
\vskip 1em
{\Large
  Profª. Juliana Betini} \\
\vskip 1em
{\Large
  Delineamento e Análise de experimentos} \\
\vskip 1em
{\Large
  Aluno: Bruno Gondim Toledo | Matrícula: 15/0167636} \\
\vskip 1em
\end{center}

\vskip 5em

\hypertarget{exercuxedcio-1-uma-engenheira-estuxe1-interessada-em-investigar-a-relauxe7uxe3o-entre-a-configurauxe7uxe3o-de-potuxeancia-de-ruxe1dio-frequuxeancia-rf-e-a-taxa-de-gravauxe7uxe3o-para-esta-ferramenta.-o-objetivo-de-um-experimento-como-este-uxe9-modelar-a-relauxe7uxe3o-entre-a-taxa-de-gravauxe7uxe3o-e-a-potuxeancia-de-rf-e-especificar-a-configurauxe7uxe3o-de-potuxeancia-que-daruxe1-uma-taxa-de-gravauxe7uxe3o-desejada.-ela-estuxe1-interessada-em-um-determinado-guxe1s-c2f6-e-uma-abertura-de-080cm-para-testar-quatro-nuxedveis-de-potuxeancia-de-rf-160-180-200-e-220-w.-ela-decidiu-testar-cinco-placas-em-cada-nuxedvel-de-potuxeancia-de-rf.-suponha-que-a-engenheira-execute-o-experimento-de-forma-aleatuxf3ria.-as-observauxe7uxf5es-que-ela-obteve-sobre-a-taxa-de-gravauxe7uxe3o-suxe3o-mostradas-na-tabela-1.}{%
\section{Exercício 1: Uma engenheira está interessada em investigar a
relação entre a configuração de potência de rádio frequência (RF) e a
taxa de gravação para esta ferramenta. O objetivo de um experimento como
este é modelar a relação entre a taxa de gravação e a potência de RF e
especificar a configuração de potência que dará uma taxa de gravação
desejada. Ela está interessada em um determinado gás (C2F6) e uma
abertura de 0,80cm para testar quatro níveis de potência de RF: 160,
180, 200 e 220 W. Ela decidiu testar cinco placas em cada nível de
potência de RF. Suponha que a engenheira execute o experimento de forma
aleatória. As observações que ela obteve sobre a taxa de gravação são
mostradas na Tabela
1.}\label{exercuxedcio-1-uma-engenheira-estuxe1-interessada-em-investigar-a-relauxe7uxe3o-entre-a-configurauxe7uxe3o-de-potuxeancia-de-ruxe1dio-frequuxeancia-rf-e-a-taxa-de-gravauxe7uxe3o-para-esta-ferramenta.-o-objetivo-de-um-experimento-como-este-uxe9-modelar-a-relauxe7uxe3o-entre-a-taxa-de-gravauxe7uxe3o-e-a-potuxeancia-de-rf-e-especificar-a-configurauxe7uxe3o-de-potuxeancia-que-daruxe1-uma-taxa-de-gravauxe7uxe3o-desejada.-ela-estuxe1-interessada-em-um-determinado-guxe1s-c2f6-e-uma-abertura-de-080cm-para-testar-quatro-nuxedveis-de-potuxeancia-de-rf-160-180-200-e-220-w.-ela-decidiu-testar-cinco-placas-em-cada-nuxedvel-de-potuxeancia-de-rf.-suponha-que-a-engenheira-execute-o-experimento-de-forma-aleatuxf3ria.-as-observauxe7uxf5es-que-ela-obteve-sobre-a-taxa-de-gravauxe7uxe3o-suxe3o-mostradas-na-tabela-1.}}

\hypertarget{tabela-1-dados-de-taxa-de-gravauxe7uxe3o-em-amin-do-experimento-de-gravauxe7uxe3o-com-plasma}{%
\subsection{Tabela 1: Dados de taxa de gravação (em A/min) do
experimento de gravação com
plasma}\label{tabela-1-dados-de-taxa-de-gravauxe7uxe3o-em-amin-do-experimento-de-gravauxe7uxe3o-com-plasma}}

\begin{longtable}[]{@{}lrrrrrr@{}}
\toprule()
potencia & 1 & 2 & 3 & 4 & 5 & total \\
\midrule()
\endhead
160 & 575 & 542 & 530 & 539 & 570 & 2756 \\
180 & 565 & 593 & 590 & 579 & 610 & 2937 \\
200 & 600 & 651 & 610 & 637 & 629 & 3127 \\
220 & 725 & 700 & 715 & 685 & 710 & 3535 \\
\bottomrule()
\end{longtable}

\newpage

\hypertarget{quais-suxe3o-as-hipuxedteses-de-interesse}{%
\subsubsection{1.1) Quais são as hipíteses de
interesse?}\label{quais-suxe3o-as-hipuxedteses-de-interesse}}

\[h_0: \tau_1 = \tau_2 = \tau_3 = \tau_4 \]

\[h_1: \exists \ \tau_i \neq \tau_j \ ; \ i \neq j\]

\[y_{ij} = \mu + \tau_i + \epsilon_{ij}\]

\hypertarget{calcule-a-estatuxedstica-do-teste-e-o-p-valor-usando-os-resultados-encontrados-na-aula-teuxf3rica-e-usando-o-software-r}{%
\subsubsection{\texorpdfstring{1.2) Calcule a estatística do teste e o
p-valor usando os resultados encontrados na aula teórica e usando o
\emph{software}
\textbf{R}}{1.2) Calcule a estatística do teste e o p-valor usando os resultados encontrados na aula teórica e usando o software R}}\label{calcule-a-estatuxedstica-do-teste-e-o-p-valor-usando-os-resultados-encontrados-na-aula-teuxf3rica-e-usando-o-software-r}}

\[ SQ_{Total} = \sum_i^a\sum_j^n(y_{ij}-\bar{y})^2 \]

\begin{longtable}[]{@{}lrrrrr@{}}
\toprule()
& Df & Sum Sq & Mean Sq & F value & Pr(\textgreater F) \\
\midrule()
\endhead
potencia & 3 & 66870.55 & 22290.18 & 66.79707 & 0 \\
Residuals & 16 & 5339.20 & 333.70 & NA & NA \\
\bottomrule()
\end{longtable}

\hypertarget{os-pressupostos-necessuxe1rios-foram-atendidos}{%
\subsubsection{1.3) Os pressupostos necessários foram
atendidos?}\label{os-pressupostos-necessuxe1rios-foram-atendidos}}

Calculando resíduos

\[ e_{ij} = y_{ij} - \hat{y}_{ij} \]

\begin{Shaded}
\begin{Highlighting}[]
\NormalTok{df\_residuos }\OtherTok{\textless{}{-}}\NormalTok{ df }\SpecialCharTok{\%\textgreater{}\%}
  \FunctionTok{mutate}\NormalTok{(}\AttributeTok{mtrat =}\NormalTok{ total}\SpecialCharTok{/}\DecValTok{5}\NormalTok{) }\SpecialCharTok{\%\textgreater{}\%}
    \FunctionTok{pivot\_longer}\NormalTok{(}\AttributeTok{cols =} \FunctionTok{c}\NormalTok{(}\StringTok{\textasciigrave{}}\AttributeTok{1}\StringTok{\textasciigrave{}}\NormalTok{,}\StringTok{\textasciigrave{}}\AttributeTok{2}\StringTok{\textasciigrave{}}\NormalTok{,}\StringTok{\textasciigrave{}}\AttributeTok{3}\StringTok{\textasciigrave{}}\NormalTok{,}\StringTok{\textasciigrave{}}\AttributeTok{4}\StringTok{\textasciigrave{}}\NormalTok{,}\StringTok{\textasciigrave{}}\AttributeTok{5}\StringTok{\textasciigrave{}}\NormalTok{),}
               \AttributeTok{values\_to =} \StringTok{\textquotesingle{}obs\textquotesingle{}}\NormalTok{,}\AttributeTok{names\_to=}\StringTok{\textquotesingle{}rep\textquotesingle{}}\NormalTok{) }\SpecialCharTok{\%\textgreater{}\%}
  \FunctionTok{select}\NormalTok{(}\DecValTok{1}\NormalTok{,}\DecValTok{3}\NormalTok{,}\DecValTok{5}\NormalTok{) }\SpecialCharTok{\%\textgreater{}\%}
  \FunctionTok{mutate}\NormalTok{(}\AttributeTok{residuo =}\NormalTok{ obs}\SpecialCharTok{{-}}\NormalTok{mtrat)}

\CommentTok{\#shapiro.test(df2$obs)}
\FunctionTok{shapiro.test}\NormalTok{(df\_residuos}\SpecialCharTok{$}\NormalTok{residuo)}
\end{Highlighting}
\end{Shaded}

\begin{verbatim}
## 
##  Shapiro-Wilk normality test
## 
## data:  df_residuos$residuo
## W = 0.93752, p-value = 0.2152
\end{verbatim}

\begin{Shaded}
\begin{Highlighting}[]
\CommentTok{\#qqplot(x=500:750,y=df2$obs)}
\CommentTok{\#qqplot(x=0:100,y=df\_residuos$residuo)}

\FunctionTok{plot}\NormalTok{(}\AttributeTok{y=}\NormalTok{df\_residuos}\SpecialCharTok{$}\NormalTok{obs,}
     \AttributeTok{x=}\NormalTok{df\_residuos}\SpecialCharTok{$}\NormalTok{residuo)}
\end{Highlighting}
\end{Shaded}

\includegraphics{aula_10_04_files/figure-latex/unnamed-chunk-1-1.pdf}

\begin{Shaded}
\begin{Highlighting}[]
\CommentTok{\#plot(df2$obs)}

\FunctionTok{qqnorm}\NormalTok{(df\_residuos}\SpecialCharTok{$}\NormalTok{residuo)}
\end{Highlighting}
\end{Shaded}

\includegraphics{aula_10_04_files/figure-latex/unnamed-chunk-1-2.pdf}

\begin{Shaded}
\begin{Highlighting}[]
\CommentTok{\#qqline(df\_residuos$residuo)}

\FunctionTok{leveneTest}\NormalTok{(}\AttributeTok{y =}\NormalTok{ df\_residuos}\SpecialCharTok{$}\NormalTok{obs,}
           \AttributeTok{group =}\NormalTok{ df\_residuos}\SpecialCharTok{$}\NormalTok{potencia,}
           \AttributeTok{data=}\NormalTok{df\_residuos)}
\end{Highlighting}
\end{Shaded}

\begin{verbatim}
## Levene's Test for Homogeneity of Variance (center = median: df_residuos)
##       Df F value Pr(>F)
## group  3  0.1959 0.8977
##       16
\end{verbatim}

Gráfico resíduo por valor ajustado

Se normal: Teste de Bartlett

Se não-normal: Teste de Levene

\hypertarget{qual-sua-conclusuxe3o-sobre-os-resultados-encontrados}{%
\subsubsection{1.4) Qual sua conclusão sobre os resultados
encontrados?}\label{qual-sua-conclusuxe3o-sobre-os-resultados-encontrados}}

\hypertarget{qual-a-proporuxe7uxe3o-da-variauxe7uxe3o-total-explicada-pelo-modelo-ajustado-no-item-1.2}{%
\subsubsection{1.5) Qual a proporção da variação total explicada pelo
modelo ajustado no item
1.2?}\label{qual-a-proporuxe7uxe3o-da-variauxe7uxe3o-total-explicada-pelo-modelo-ajustado-no-item-1.2}}

\[r^2 = \frac{SQ_{tratamentos}}{SQ_{total}}\]

0.9260598

\hypertarget{se-a-hipuxf3tese-nula-for-rejeita-quais-potuxeancias-de-ruxe1dio-frequuxeancia-diferem-entre-si-apresente-as-hipuxf3teses-que-seruxe3o-testadas-e-a-estatuxedstica-do-teste}{%
\subsubsection{1.6) Se a hipótese nula for rejeita, quais potências de
rádio frequência diferem entre si? Apresente as hipóteses que serão
testadas e a estatística do
teste}\label{se-a-hipuxf3tese-nula-for-rejeita-quais-potuxeancias-de-ruxe1dio-frequuxeancia-diferem-entre-si-apresente-as-hipuxf3teses-que-seruxe3o-testadas-e-a-estatuxedstica-do-teste}}

Teste de Fisher, teste de Tuckey

\begin{verbatim}
## [1] 551.2
\end{verbatim}

\begin{verbatim}
## [1] 587.4
\end{verbatim}

\begin{verbatim}
## [1] 625.4
\end{verbatim}

\begin{verbatim}
## [1] 707
\end{verbatim}

\begin{verbatim}
##   Tukey multiple comparisons of means
##     95% family-wise confidence level
## 
## Fit: aov(formula = obs ~ potencia, data = df2)
## 
## $potencia
##          diff        lwr       upr     p adj
## 180-160  36.2   3.145624  69.25438 0.0294279
## 200-160  74.2  41.145624 107.25438 0.0000455
## 220-160 155.8 122.745624 188.85438 0.0000000
## 200-180  38.0   4.945624  71.05438 0.0215995
## 220-180 119.6  86.545624 152.65438 0.0000001
## 220-200  81.6  48.545624 114.65438 0.0000146
\end{verbatim}

\hypertarget{considere-que-antes-de-realizar-o-experimento-a-engenheira-tinha-a-suposiuxe7uxe3o-de-diferenuxe7a-entre-as-potuxeancias-mais-baixas-e-as-potuxeancias-mais-altas.-construa-um-conjunto-de-contrastes-ortogonais-a-partir-dessa-informauxe7uxe3o.-apresente-as-hipuxf3teses-que-seruxe3o-testadas-as-conclusuxf5es-e-a-estatuxedstica-de-teste-considerada.}{%
\subsubsection{1.7) Considere que antes de realizar o experimento, a
engenheira tinha a suposição de diferença entre as potências mais baixas
e as potências mais altas. Construa um conjunto de contrastes ortogonais
a partir dessa informação. Apresente as hipóteses que serão testadas, as
conclusões e a estatística de teste
considerada.}\label{considere-que-antes-de-realizar-o-experimento-a-engenheira-tinha-a-suposiuxe7uxe3o-de-diferenuxe7a-entre-as-potuxeancias-mais-baixas-e-as-potuxeancias-mais-altas.-construa-um-conjunto-de-contrastes-ortogonais-a-partir-dessa-informauxe7uxe3o.-apresente-as-hipuxf3teses-que-seruxe3o-testadas-as-conclusuxf5es-e-a-estatuxedstica-de-teste-considerada.}}

\[C_1 \begin{cases}
            H_0 : \mu_1 + \mu_2 - \mu_3 - \mu_4 = 0 \\
            H_1 : \mu_1 + \mu_2 - \mu_3 - \mu_4 \neq 0
        \end{cases}\]

\[C_2 \begin{cases}
            H_0 : \mu_1 - \mu_2 = 0 \\
            H_1 : \mu_1 - \mu_2 \neq 0
        \end{cases}\]

\[C_3 \begin{cases}
            H_0 : \mu_3 - \mu_4 = 0 \\
            H_1 : \mu_3 - \mu_4 \neq 0
        \end{cases}\]

Para cada \(c_i\), calcular a estatística de teste

\[t_0 = \frac{\sum_{i=1}^ac_i\bar{y}_{i.}}{\frac{QM_{res}}{n}\sum_{i=1}^ac_i^2}\]

\hypertarget{calcule-a-probabilidade-do-erro-tipo-ii-para}{%
\subsubsection{1.8) Calcule a probabilidade do erro tipo II
para:}\label{calcule-a-probabilidade-do-erro-tipo-ii-para}}

\[\mu_1 = 575;\\ \mu_2 = 600;\\ \mu_3 = 650;\\ \mu_4 = 675;  \ e \\ \sigma = 25.\]

\[\begin{cases}
Erro \ tipo \ I  &: P(Rejeitar \ H_0         &| H_0 \ verdadeira)\\
Erro \ tipo \ II &: P(Não \ rejeitar \ H_0 \ &| H_0 \ falso)\\
\end{cases}\]

\hypertarget{qual-deve-ser-o-nuxfamero-de-repetiuxe7uxf5es-no-experimento-para-que-o-erro-seja-menor-que-0.01}{%
\subsubsection{1.9) Qual deve ser o número de repetições no experimento
para que o erro seja menor que
0.01\%?}\label{qual-deve-ser-o-nuxfamero-de-repetiuxe7uxf5es-no-experimento-para-que-o-erro-seja-menor-que-0.01}}

\end{document}
